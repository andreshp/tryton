%%%%%%%%%%%%%%%%%%%%%%%%%%%%%%%%%%%%%%%%%%%%%%%%%%%%%%%%%%%%%%%%%%%%%%%%%%%%%%%%%%%%%%%%%%%%%%%%%%%%%%
% Plantilla básica de Latex en Español.
%
% Autor: Andrés Herrera Poyatos (https://github.com/andreshp)
%
% Es una plantilla básica para redactar documentos. Utiliza el paquete fancyhdr para darle un
% estilo moderno pero serio.
%
% La plantilla se encuentra adaptada al español.
%
%%%%%%%%%%%%%%%%%%%%%%%%%%%%%%%%%%%%%%%%%%%%%%%%%%%%%%%%%%%%%%%%%%%%%%%%%%%%%%%%%%%%%%%%%%%%%%%%%%%%%%

%-----------------------------------------------------------------------------------------------------
%	INCLUSIÓN DE PAQUETES BÁSICOS
%-----------------------------------------------------------------------------------------------------

\documentclass{article}

\usepackage{lipsum}                     % Texto dummy. Quitar en el documento final.

%-----------------------------------------------------------------------------------------------------
%	SELECCIÓN DEL LENGUAJE
%-----------------------------------------------------------------------------------------------------

% Paquetes para adaptar Látex al Español:
\usepackage[spanish,es-noquoting, es-tabla, es-lcroman]{babel} % Cambia
\usepackage[utf8]{inputenc}                                    % Permite los acentos.
\selectlanguage{spanish}                                       % Selecciono como lenguaje el Español.

%-----------------------------------------------------------------------------------------------------
%	SELECCIÓN DE LA FUENTE
%-----------------------------------------------------------------------------------------------------

% Fuente utilizada.
\usepackage{courier}                    % Fuente Courier.
\usepackage{microtype}                  % Mejora la letra final de cara al lector.

%-----------------------------------------------------------------------------------------------------
%	ESTILO DE PÁGINA
%-----------------------------------------------------------------------------------------------------

% Paquetes para el diseño de página:
\usepackage{fancyhdr}               % Utilizado para hacer títulos propios.
\usepackage{lastpage}               % Referencia a la última página. Utilizado para el pie de página.
\usepackage{extramarks}             % Marcas extras. Utilizado en pie de página y cabecera.
\usepackage[parfill]{parskip}       % Crea una nueva línea entre párrafos.
\usepackage{geometry}               % Asigna la "geometría" de las páginas.

% Se elige el estilo fancy y márgenes de 3 centímetros.
\pagestyle{fancy}
\geometry{left=3cm,right=3cm,top=3cm,bottom=3cm,headheight=1cm,headsep=0.5cm} % Márgenes y cabecera.
% Se limpia la cabecera y el pie de página para poder rehacerlos luego.
\fancyhf{}

% Espacios en el documento:
\linespread{1.1}                        % Espacio entre líneas.
\setlength\parindent{0pt}               % Selecciona la indentación para cada inicio de párrafo.

% Cabecera del documento. Se ajusta la línea de la cabecera.
\renewcommand\headrule{
	\begin{minipage}{1\textwidth}
	    \hrule width \hsize
	\end{minipage}
}

% Texto de la cabecera:
\lhead{\docauthor}                          % Parte izquierda.
\chead{}                                    % Centro.
\rhead{\subject \ - \doctitle}              % Parte derecha.

% Pie de página del documento. Se ajusta la línea del pie de página.
\renewcommand\footrule{
\begin{minipage}{1\textwidth}
    \hrule width \hsize
\end{minipage}\par
}

\lfoot{}                                                 % Parte izquierda.
\cfoot{}                                                 % Centro.
\rfoot{Página\ \thepage\ de\ \protect\pageref{LastPage}} % Parte derecha.

%-----------------------------------------------------------------------------------------------------
%   CONFIGURACIÓN DE LOS ENLACES
%-----------------------------------------------------------------------------------------------------

\usepackage{hyperref}

\hypersetup{
    colorlinks   = true,   % Quita las cajas y añade un color al texto.
    % Tipos de enlaces cuyo color se puede configurar:
    linkcolor    = cyan,        % Por defecto red
    anchorcolor  = gray,        % Por defecto black
    citecolor    = magenta,     % Por defecto green
    filecolor    = red,         % Por defecto cyan
    menucolor    = green,       % Por defecto red
    runcolor     = red,         % Por defecto cyan
    urlcolor     = cyan,        % Por defecto magenta
    allcolors    = cyan,        % Si se usa esta opción todos los enlaces de cualquier tipo adquieren el color dado.
}

%-----------------------------------------------------------------------------------------------------
%   CONFIGURACIÓN DE LA INCLUSIÓN DE CÓDIGO
%-----------------------------------------------------------------------------------------------------

% Paquetes para la insercción y el resaltado del código:
\usepackage[usenames,dvipsnames]{color} % Permite crear colores propios. Utilizado para el bg de Minted.
\usepackage{minted}                     % Insercción y resaltado de código con Minted.

\usemintedstyle{autumn}                      % Se elige el estilo para minted.
\definecolor{bg}{rgb}{0.95,0.95,0.95}        % Se define el color bg usado para bgcolor de Minted.
\renewcommand\listingscaption{Código}        % Se redefine el nombre dado a un bloque de código.
\renewcommand\listoflistingscaption{Códigos} % Se redefine el nombre dado a la lista de códigos.

%-----------------------------------------------------------------------------------------------------
%	PORTADA
%-----------------------------------------------------------------------------------------------------

% Elija uno de los siguientes formatos.
% No olvide incluir los archivos .sty asociados en el directorio del documento.
\usepackage{title1}
%\usepackage{title2}

%-----------------------------------------------------------------------------------------------------
%	TÍTULO, AUTOR Y OTROS DATOS DEL DOCUMENTO
%-----------------------------------------------------------------------------------------------------

% Título del documento.
\newcommand{\doctitle}{Tryton}
% Subtítulo.
\newcommand{\docsubtitle}{}
% Fecha.
\newcommand{\docdate}{1 \ de \ Enero \ de \ 2017}
% Asignatura.
\newcommand{\subject}{Diseño y desarrollo de sistemas de información}
% Autor.
\newcommand{\docauthor}{A. Herrera, M. Morales, M. Ruiz}
\newcommand{\docaddress}{Universidad de Granada}
\newcommand{\docemail}{andreshp9@gmail.com}

%-----------------------------------------------------------------------------------------------------
%	RESUMEN
%-----------------------------------------------------------------------------------------------------

% Resumen del documento. Va en la portada.
% Puedes también dejarlo vacío, en cuyo caso no aparece en la portada.
%\newcommand{\docabstract}{}
\newcommand{\docabstract}{En este texto puedes incluir un resumen del documento. Este informa al lector sobre el contenido del texto, indicando el objetivo del mismo y qué se puede aprender de él.}

\begin{document}

\maketitle

%-----------------------------------------------------------------------------------------------------
%	ÍNDICE
%-----------------------------------------------------------------------------------------------------

% Profundidad del Índice:
%\setcounter{tocdepth}{1}

\newpage
\tableofcontents
\newpage

%-----------------------------------------------------------------------------------------------------
%	SECCIÓN 1
%-----------------------------------------------------------------------------------------------------

\section{Introducción y Nota Histórica}

Tryton es una plataforma informática general de alto nivel basada en el framework Tiny 4.2. Tiny 4.2 era una solución de negocios ERP perteneciente a la entonces empresa TinyERP, que pasó a llamarse OpenERP y actualmente Odoo S.A. Dicha empresa desarrolla un sistema ERP Commercial Open Source Proyect (Proyecto de Código abierto comercial), esto es, el código es libre, pero es la empresa la que selecciona que mejoras se incorporan al proyecto. 

Empezaron entonces a desarrollarse numerosas mejoras por parte de las empresasc colaboradoras que OpenERP SA decicdió no integrar, provocando el desarrollo paralelo de Tryton como un fork de Tiny 4.2 suponiendo una alternativa totalmente libre, suponiendo un proyecto totalmente desarrollado por la comunidad. Su primera versión apareció en Noviembre de 2008.\url{http://openerpspain.com/que-es-tryton/}

Por tanto Tryton supone un framework con características similares a las que ofrecía Tiny, esto es, está desarrollado en Python y usa como base de datos principalmente PostgreSQL. Además, dicha plataforma de alto nivel está diseñada en tres capas (cliente Tryton, servidor Tryton y Base de Datos), ofreciendo una amplia funcionalidad mediante módulos. Los módulos presentes actualmente engloban las siguientes categorías:\url{https://www.tryton.org/es/}

\begin{itemize}
    \item Contabilidad
    \item Facturación
    \item Gestión de ventas
    \item Gestión de compras
    \item Contabilidad analítica
    \item Gestión de inventario
    \item Fabricación: Manufacturing Resource Planning (MRP)
    \item Gestión de proyectos
    \item Gestión de iniciativas y oportunidades
\end{itemize}

Dichas categorias a su vez incluyen una amplia gama de módulos (Todos los módulos pueden consultarse en \url{https://pypi.python.org/pypi?:action=browse&show=all&c=551})

%-----------------------------------------------------------------------------------------------------
%   SECCIÓN 2
%-----------------------------------------------------------------------------------------------------

\section{Descripción de la instalación}

%-----------------------------------------------------------------------------------------------------
%   SECCIÓN 3
%-----------------------------------------------------------------------------------------------------

\section{Funcionalidad que ofrece}

Las funcionalidades que aporta son las siguientes:

persistencia de datos, extensa modularidad, administración de usuarios (autentificación, control detallado de acceso a los datos, manejo concurrente a recursos), flujos de trabajo y motores de reportes, servicios web e internacionalización.

\url{https://www.tryton.org/es/}

%-----------------------------------------------------------------------------------------------------
%   SECCIÓN 4
%-----------------------------------------------------------------------------------------------------

\section{Ventajas e inconvenientes encontrados durante la prueba}
Ventajas:
\begin{itemize}
    \item Su estructura es mucho más parecida a la de Django, presentando una mayor escalabilidad que su alternativa en OpenERP. Esto facilita el desarrollo de módulos específicos a cada cliente
    \item Facilidad de migración entre versiones.
\end{itemize}

Problemas:
\begin{itemize}
    \item Numerosos módulos que dificultan su uso--> cada cliente puede instalar tan solo los módulos que necesita
    \item La total libertad para desarrollar dificulta un desarrollo coherente y ordenado del proyecto--> a partir de 2012 las empresas Zikzakmedia y NaN-Tic pasan a estar detrás de Tryton en la entonces creada fundación Tryton, coordinando el desarrollo del proyecto, pero sin tomar decisiones sobre la funcionalidad a añadir.(\url{http://www.zikzakmedia.com/es/tryton.html})
\end{itemize}

%-----------------------------------------------------------------------------------------------------
%   SECCIÓN 5
%-----------------------------------------------------------------------------------------------------

\section{Empresas que le dan soporte}

Tryton presenta un gran soporte comunitario, pero además es respaldado por varias compañías que proveen servicios profesionales:
B2CK, incore, Leuchter Open Source Solutions, MBSolutions, NaN-tic, OPDevel, Openlabs Technologies  \& Consulting (P) Limited, SISalp, Soluciones de Inteligencia de Mercados, Thymbra, 
Virtual Things, Zikzakmedia.(\url{http://www.tryton.org/es/servicios.html})

%-----------------------------------------------------------------------------------------------------
%   SECCIÓN 6
%-----------------------------------------------------------------------------------------------------

\section{Empresas que lo han implantado}
Como ejemplo de empresas que han implantado Tryton podemos nombrar Zona Franca, ColomTel,10ENLACE, MOTOPROMET,Turesandes,H\&B Negocios y Suministros, MECATRONIC, Ibo Consultorias, etc.
(\url{http://www.presik.com})


%-----------------------------------------------------------------------------------------------------
%	SECCIÓN 2
%-----------------------------------------------------------------------------------------------------

\section{Segunda Sección}

	\subsection{Primera subsección}

    \begin{listing}
        \caption{Estructura que representa una Página en Linux}
        \begin{minted}[bgcolor=bg]{c}
            #include <linux/mm_types.h>

            struct page {
                unsigned long flags;
                atomic_t _count;
                atomic_t _mapcount;
                unsigned long private;
                struct address_space *mapping;
                pgoff_t index;
                struct list_head lru;
                void *virtual;
            }
        \end{minted}
        \label{listing:page}
    \end{listing}

	\subsection{Segunda subsección}

		\lipsum[2]

\end{document}
